\section{From classical to quantum walks}

Classical random walks describe a stohastic process

Tool in mathematics for modeling discrete time and discrete space stochastic processes.

Used in for directly modeling:
\begin{itemize}
\item Finance: Stock price movement
\item Natural Language Processing Algorithms
\item Engineering Physics: Brownian motion
\item Biology: DNA evolution models, population dynamics, modeling disease outbreaks, epidemic modeling.

\end{itemize}

- PAGE RANK IS RANDOM WALK

Used in algorithms, utilizing randomness:
\begin{itemize}
\item Markov chain Monte Carlo: Sampling from a probability distribution. Construct markov chain that has the desired equilibrium distribution, then sample by recording states from the chain.
\item This Metropolis–Hastings algorithm: sampling distribution from which direct sampling is difficult, then approximate the distribution or compute an integral (e.g. expected value).

\end{itemize}

Classical random walks on graphs can be defined using Markov-chains. Markov-chains
are well explained in Leo Breiman's book on Probability\cite{breiman_probability_1992}.

\definition{\textbf{Markov-chain}} A Markov-chain is a process consisting of random variables