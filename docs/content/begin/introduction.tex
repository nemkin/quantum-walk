\chapter{Introduction}

\section{Classical random walks on graphs}

Classical random walks are used to describe stochastic processes and many real life sciences rely on these methods. For example stock proce movement prediction in finance, natural language processing algorithms, describing brownian motion in engineering physics, and various applications in biology and bioinformatics, such as DNA evolution models, population dynamics, modeling disease outbreaks, epidemic modeling ...etc. There are also algorithms, where stochastic processes are not directly present, but introduced as a way to deal with large quantities of data. Most notably, Google's Page Rank algorithm utilizes classical random walks on the large sized graph of the internet, to score documents on how good of an information source they are or how well they link to other information sources. Other algorithms include Markov chain Monte Carlo, for sampling from a probability distribution, difficult to directly model. Instead, they construct a stohastic process which's equilibrium distribution is the desired one and sample this by repeatedly recording states from the chain. The Metropolis Hashtings algorithm is similar to this, and is used to approximate the distribution or compute an integral (e.g expected value).

\unsure{Itt az lenne a célom hogy eladjam a véletlen sétákat, hogy ezek nagyon fontos algoritmusok. A Google PageRank egy nagyon jó példa.}

In conclusion, there are many important applications for classical random walks.

\section{From classical to quantum}

There are many properties of quantum walks that make them differ from their classical counterparts. While in classical walks, in N steps the walks reaches a distance of $O(\sqrt{N})$, in quantum, it is a lot faster, $O(N)$. Quantum walks have inherent interferences, where some amplitudes can amplify, while others can be destrutive to each other and diminish.

The

\section{Applications}

A bevezető tartalmazza a diplomaterv-kiírás elemzését, történelmi előzményeit, a feladat indokoltságát (a motiváció leírását), az eddigi megoldásokat, és ennek tükrében a hallgató megoldásának összefoglalását.

A bevezető szokás szerint a diplomaterv felépítésével záródik, azaz annak rövid leírásával, hogy melyik fejezet mivel foglalkozik.
