\pagenumbering{roman}
\setcounter{page}{1}

\selecthungarian

%----------------------------------------------------------------------------
% Abstract in Hungarian
%----------------------------------------------------------------------------
\chapter*{Kivonat}\addcontentsline{toc}{chapter}{Kivonat}

Az utóbbi években egyre nagyobb figyelem összpontosul a kvantuminformatikára. Olyan globális vállalatok, mint az IBM, a Google, a Microsoft és az Amazon jelentős összegeket fektetnek kutatásba, hardver- és szoftverfejlesztésekbe ezen a területen, míg az Európai Unió és Magyarország számos olyan programot indított, melyek a kvantuminformatikai kutatások fellendítését célozzák meg.

A jelenlegi kvantumszámítógépekben elérhető qubitek (kvantumbitek) mennyisége még csekély, de sokan úgy vélik hogy a jövőben ez a szám növekedni fog. Az első olyan, a gyakorlatban is hasznos kvantumalgoritmusok, amiket ezeken a processzorokon futtatni tudunk majd, várhatóan azok lesznek, melyek takarékosan bánnak a rendelkezésre álló qubitekkel. A kvantumséta, mely a klasszikus véletlen bolyongás általánosítása kvantumos esetben, pontosan ilyen algoritmus. Mivel a qubit igénye a gráf csúcsszámában logaritmikus, így ez egy érdekes módszernek ígérkezik akár a közeljövőre nézve is. A kvantumséták erejét bizonyítja, hogy a Grover keresés (mely több kvantumalgoritmus alapját képezi) is értelmezhető ezek egy speciális fajtájának.

Dolgozatomban leírom a kvantumséták matematikai alapjait, részletezve a megvalósítás szempontjából fontos pontokat, melyek a szakirodalomban kisebb hangsúllyal szerepelnek. Ismertetem az általam írt szimulátor program architekturális felépítését és működését, továbbá a futtatott szimulációim eredményeit.

A szimulátor programot Python 3 nyelven írtam, a Stratégia tervezési minta alapján. A szakirodalomban tipikusan használt gráfokat beépítetten támogatja, melyek kombinálásával tetszőlegesen bonyolult reguláris gráf előállítható, ez az előállítás képezi a kvantumséta alapját is. A szoftver reguláris gráfokon történő kvantum és klasszikus séták szimulációját teszi lehetővé, az eredményekről pedig egy részletes report fájlt generál. A kvantumos séták esetében a séta tulajdonságai a valószínűségek generálásához felhasznált érmétől is függnek, melyet többféleképpen is lehet definiálni. A program beépítetten tartalmazza az Hadamard-, a Grover- és a Fourier-érméket, de felépítéséből adódóan könnyen bővíthető tetszőleges érmével is.

Szimulációim segítségével összehasonlítottam a klasszikus és a kvantum séták viselkedését, továbbá kimutattam az elméleti szakirodalom alapján elvárt kvantumos jellegzetességeket, az Hadamard-séta ballisztikus természetét és a kvantumséták ciklikus tulajdonságát.


\vfill
\selectenglish


%----------------------------------------------------------------------------
% Abstract in English
%----------------------------------------------------------------------------
\chapter*{Abstract}\addcontentsline{toc}{chapter}{Abstract}

In recent years, there has been an increasing focus on quantum informatics. Influential global companies such as IBM, Google, Microsoft, and Amazon have invested significant amounts into studying and developing hardware and software for this sector, while the European Union and Hungary have launched several programs to accelerate quantum research.

Current technology is yet to produce a significant number of qubits (quantum bits) in a quantum processor, but many believe the amount will increase over the years. The first practical quantum algorithms to be run on these processors are likely to be the ones that use qubits sparingly. Quantum walking, the generalized version of classical random walking, is exactly this kind of algorithm. The number of qubits required to run a quantum walk on a graph is logarithmic in the number of vertices, making it a promising technique for the near future. Furthermore, Grover's search algorithm (a basis for many quantum algorithms) can be viewed as a special case of quantum walks, which illustrates the potential power of this method.

In my dissertation, I present the mathematical framework for quantum walks, detailing the points critical for implementation, which are given less emphasis in the literature. I describe the architecture and capabilities of the simulator program I have written and the conclusions of the simulations I have run.

I developed the software using Python 3, based on the Strategy design pattern. It supports graphs commonly found in the literature while also providing a method for combining them, facilitating experimentation on several kinds of regular graphs. This composition is also the foundation of the quantum walk. It can simulate classical and quantum walks on the same graphs and produce a report file detailing the results. In the quantum case, the characteristics of the walk are also dependent on the type of coin used to generate the probabilities, which can be defined in several ways. The program includes the Hadamard, Grover, and Fourier coins and can easily be extended with others.

Running several simulations, I compared the behavior of classical and quantum walks and demonstrated the quantum characteristics expected from the theoretical literature, the ballistic nature of the Hadamard walk, and the cyclic property of quantum walks.


\vfill
\selectthesislanguage

\newcounter{romanPage}
\setcounter{romanPage}{\value{page}}
\stepcounter{romanPage}